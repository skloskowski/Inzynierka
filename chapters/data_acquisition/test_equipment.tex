\section{Procedura i zestawy danych}

    Do wykonania fotografii do badania i testowania aplikacji zastosowano następujący sprzęt:

    \begin{itemize}
        \item Lustrzanka Nikon D800 - z modyfikacją pozwalającą na zwiększenie sygnału docierającego do matrycy z zakresu bliskiego IR.
        \item Obiektyw Samyang 135mm F2.0.
        \item Głowica paralaktyczna Sky-Watcher Star Adventurer z przeciwwagą.
        \item Statyw fotograficzny.
        \item Ocieplacz przedniego elementu obiektywu.
    \end{itemize}

    Procedura wykorzystywana do przechwytywania zdjęć obiektów głębokiego nieba wyglądała następująco. 

    \begin{enumerate}
        \item Przeniesienie sprzętu fotograficznego wraz z zasileniem w miejsce niedostępne dla okolicznych zanieczyszczeń świetlnych.
        \item Założenie głowicy na statyw fotograficzny oraz dołączenie aparatu z obiektywem do głowicy z przeciwwagą.
        \item Wstępne ustawienie prawidłowego wyrównania z biegunem głowicy.
        \item Ustawienie aparatu z przeciwwagą do zbalansowanego stanu.
        \item Ustawienie aparatu na określoną część nieba.
        \item Nałożenie poprawek na wyrównanie z biegunem.
        \item Końcowe ustawienie prawidłowych opcji aparatu. Znalezienie ostrości aparatu na gwiazdach.
        \item Wykonanie fotografii testowych.
        \item Poprawki po fotografii testowej.
        \item Włączenie aparatu na wykonanie odpowiedniej ilości zdjęć.
        \item Po zakończeniu głównej części wykonywano zdjęcia kalibracyjne (opcjonalnie).
    \end{enumerate}

    Proces ten zapewnił, że zdjęcia będą w odpowiedniej jakości. Przy nieprawidłowym wyrównaniu głowicy z biegunem lub innych niedopracowaniach mechanicznych programy obsługujące obróbkę zdjęć mogą nie przetworzyć zdjęć prawidłowo.

    Korzystając z wyżej wymienionej procedury przygotowano dwa główne zestawy zdjęć, które zostaną następnie podzielone do badań. Dla jednoznacznego nazwania obiektów korzystano z katalogu NGC (\textit{New General Catalogue}) w najnowszej udostępnionej wersji \cite{ngc}. Podzielone zestawy zostały nazwane zgodnie z podanym schematem: Zestaw\textit{numer z katalogu NGC}\_\textit{czas integracji zdjęcia [min]}. W tabeli \ref{tab:objectSets} opisano podzielone zestawy zdjęć. Określenie jasności nieba w skali Bortle'a zostało wykonane korzystając ze strony \cite{lightpollutionmap} i następnie zaokrąglone do pełnej wartości. Czasem integracji określa się całkowity czas naświetlania wszystkich połączonych fotografii.

    Zestawy rozpoczynające się od NGC224 oznaczają fotografie obiektu NGC224 znanego jako galaktyka Andromedy znajdującym się w gwiazdozbiorze Andromedy. Fotografie obiektu zostały wykonane pod niebem o jasności 3 w skali Bortle'a. Zostało wykonanych 30 fotografii, każda z ekspozycji trwała 120 sekund. Przesłona obiektywu została ustawiona na F2.0. ISO aparatu fotograficznego zostało ustawione na 800. Nie wykonano zdjęć kalibracyjnych.
    
    Zestawy rozpoczynające się od NGC1499 oznaczają fotografie obiektu NGC1499 znanego jako mgławica Kalifornia znajdującym się w gwiazdozbiorze Perseusza. Fotografie obiektu zostały wykonane pod niebem o jasności 6 w skali Bortle'a. Zostało wykonanych 60 fotografii, każda z ekspozycji trwała 60 sekund. ISO aparatu fotograficznego zostało ustawione na 800. Przesłona obiektywu została ustawiona na F2.0. Dodatkowo wykonano pełny zestaw zdjęć kalibracyjnych.

\begin{table}[h!] 
\centering
\begin{tabularx}{\textwidth}{X X X X} 
\toprule
Nazwa & Obiekt & Czas integracji [s] & Zdjęcia kalibracyjne \\
\midrule
Zestaw224\_10  & NGC224  & 600 & NIE \\
Zestaw224\_20  & NGC224  & 1200 & NIE \\
Zestaw224\_30  & NGC224  & 1800 & NIE \\
Zestaw224\_40  & NGC224  & 2400 & NIE \\
Zestaw224\_50  & NGC224  & 3000 & NIE \\
Zestaw224\_60  & NGC224  & 3600 & NIE \\
Zestaw1499\_10 & NGC1499 & 600 & TAK \\
Zestaw1499\_20 & NGC1499 & 1200 & TAK \\
Zestaw1499\_30 & NGC1499 & 1800 & TAK \\
Zestaw1499\_40 & NGC1499 & 2400 & TAK \\
Zestaw1499\_50 & NGC1499 & 3000 & TAK \\
Zestaw1499\_60 & NGC1499 & 3600 & TAK \\
\bottomrule
\end{tabularx}
\caption{Opis podzielonych zestawów fotografii obiektów głębokiego nieba} \label{tab:objectSets}
\end{table}