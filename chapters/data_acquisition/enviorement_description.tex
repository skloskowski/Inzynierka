\section{Opis otoczenia i montażu}

    Zebranie przydanych danych w trakcie sesji fotograficznej wymaga odpowiednich warunków obserwacyjnych. Najważniejszymi aspektami jest stan pogody w miejscu fotografowania, jasność nieba oraz jakość widzenia. Sesje zostały przeprowadzone podczas bezchmurnych godzin dnia podczas nocy astronomicznej. Nocą astronomiczną określa się porę dnia, podczas której Słońce jest poniżej 15 stopni pod horyzontem. W celu ograniczenia osiadania się wilgoci na przedniej części obiektywu sesje zostały wykonywane w warunkach poniżej 80 procent wilgotności powietrza zgodnie z lokalną prognozą pogody. Dodatkowym zabezpieczeniem było użycie ocieplacza przedniego elementu obiektywu.

    W celu określenia jasności nieba wielu amatorów astronomii odnosi się to tak zwanej Skali Bortle'a. Treść oryginalnego artykułu można znaleźć na stronie \cite{Bortle2006}. Skala Bortle'a to skala, która określa jasność nocnego nieba w od 1 do 9, gdzie 1 to najciemniejsze nocne nieba a 9 to najjaśniejsze. Skala opisuje jakie są najciemniejsze elementy nocnego nieba, które człowiek jest w stanie zaobserwować gołym okiem.

    Kolejnym elementem ważnym dla astrofotografii jest widzenie astronomiczne. Widzenie astronomiczne jest miarą określającą poziom deformacji obrazu spowodowanego turbulencjami powietrza w atmosferze \cite{DTIC2001}. Obiekty takie jak ELT (\textit{Extremely Large Telescope}) czy inne profesjonalne teleskopy są budowane na wysokościach nad poziomem morza ograniczających wpływ gęstej atmosfery Ziemi na jakość badań.

    W celu zminimalizowania wpływu szumu odczytu oraz zmniejszenia ilości danych do przetworzenia stosuje się śledzenie i sterowanie zestawu optycznego aparatu. Żeby móc prawidłowo wykorzystać narzędzia służące do śledzenia i sterowania, zestaw optyczny trzeba ustawić równolegle do osi obrotu Ziemii.  Takie wyrównanie z biegunem (\textit{ang. polar aligmnent}) w sprzęcie dla początkujących przeprowadza się samemu korzystając z teleskopu \cite{staradventurer}, jednak proces ten w głowicach wyższej klasy został zautomatyzowany. 
    
    Śledzenie w kontekście astrofotografii odnosi się do śledzenia ruchu gwiazd na nocnym niebie - pozwala to na wykonanie fotografii o dłuższym czasie naświetlania. Bez śledzenia, fotografie o odpowiednio długim czasie naświetlania, będą powodowały rozmycie się gwiazd na fotografii. Sterowanie tym śledzeniem przy zastosowaniu drugiego teleskopu pozwala na wprowadzanie odpowiednich korekt do głowicy tym samym zwiększając precyzję śledzenia.