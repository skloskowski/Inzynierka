\section{Zarys pracy}

Niniejsza praca skupiająca się na implementacji aplikacji desktopowej do przetwarzania obrazów obiektów głębokiego nieba została podzielona na siedem rozdziałów. W rozdział pierwszy wprowadza czytelnika w tematykę, cele tworzonej aplikacji oraz strukturę pracy.
Drugi rozdział zawiera omówienie rozwiązań obecnych na rynsku oraz przybliża prace naukowe związane z przetwarzaniem obrazów astronomicznych. Trzeci rozdział przybliza pojęcia związane z przetwarzaniem obrazów astronomicznych, takie jak kalibracja,
rejestracja oraz stacking opisując algorytmy wykorzystane do ich realizacji, wyjaśnia również w jakim celu stosuje się przetwarzanie zdjęć i jak wpływa to na teoretyczną jakość obrazu końcowego. Rozdział czwarty opisuje wymagania, które należy spełnić 
w trakcie implmentacji aplikacji oraz technologie wykorzystane do jej stworzenia. Opisane są również moduły aplikacji ich funkcjonalności, interfejs użytkownika oraz jak wygląda rzeczywista implementacja najważniejszych elementów kodu.
W rozdziale piątym został opisany proces zbierania danych wraz z procedurą testową oraz opis warunków w jakich wykonano fotografie. Rozdział szósty przedstawia metodykę przeprowadzonych testów wydajnościowych oraz jakościowych wraz z wynikami przeprowadzonych
badań oraz analizą otrzymanych rezultatów. Ostatni rozdział siódmy zawiera podsumowanie i wnioski końcowe dotyczące przeprowadzonej pracy oraz propozycje dalszych kierunków rozwoju aplikacji oraz dodatkowych badań.
