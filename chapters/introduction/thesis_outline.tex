\section{Zarys pracy}

Niniejsza praca została podzielona na siedem rozdziałów. Rozdział pierwszy wprowadza czytelnika 
w tematykę, cele tworzonej aplikacji oraz strukturę pracy. Drugi rozdział zawiera omówienie rozwiązań obecnych na rynku oraz przybliża prace naukowe związane z przetwarzaniem obrazów astronomicznych. 
Trzeci rozdział przybliża pojęcia związane z przetwarzaniem obrazów astronomicznych, takie jak kalibracja, rejestracja oraz stacking, opisując algorytmy wykorzystane do ich realizacji. Wyjaśnia również 
w jakim celu stosuje się przetwarzanie zdjęć i jak wpływa to na teoretyczną jakość obrazu końcowego. Rozdział czwarty opisuje wymagania, które należy spełnić w trakcie implmentacji aplikacji oraz 
technologie wykorzystane do jej stworzenia. Opisane są również moduły aplikacji, ich funkcjonalności, interfejs użytkownika oraz jak wygląda rzeczywista implementacja najważniejszych elementów kodu.
W rozdziale piątym został opisany proces zbierania danych wraz z procedurą testową oraz warunki w jakich wykonano fotografie. Rozdział szósty przedstawia metodykę przeprowadzonych testów 
wydajnościowych i jakościowych, wraz z wynikami przeprowadzonych badań i analizą otrzymanych rezultatów. Rozdział siódmy zawiera podsumowanie i wnioski końcowe dotyczące przeprowadzonej 
pracy. Dodatkowo przedstawiono propozycje dalszych kierunków rozwoju aplikacji oraz dodatkowych badań.
