\section{Cel pracy}

Obecnie na rynku dostępne są komercyjne i darmowe narzędzia do przetwarzania zdjęć głębokiego nieba. Wiele z tych aplikacji jest skierowana jednak do zaawansowanych użytkowników.
Oferuje obłusgę wyłącznie przez interfejs graficzny nie pozwalając na automatyzację procesuu przetwarzania zdjęć. Nie posiadają również przystosować do wykonywania badań wydajnościowch i jakościowych.
Celem niniejszej pracy jest stworzenie aplikacji desktopowej, która umożliwi przetwarzanie zdjęć głębokiego nieba z wykorzystaniem podstawowych technik obróbki obrazów astronomicznych. Głównymi zaletami będzie
prosta obsługa dla użytkowników początkujących i średnio-zaawansowanych, użytkwanie za pomocą linii komend i możliwość przeprowadzania badań wydajnościowych i jakościowych.
Dodatkowo aplikacja będzie dopuszczała prostą rozbudowę o dodatkowe techniki obróbki obrazów, utworzenie interfejsu graficznego aplikacji oraz dodanie nowych funkcjonalności badawczych w przyszłości.