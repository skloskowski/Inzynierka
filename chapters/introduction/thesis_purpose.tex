\section{Cel pracy}

Obecnie na rynku dostępne są różne narzędzia do przetwarzania zdjęć głębokiego nieba - zarówno komercyjne, jak i darmowe. Wiele z tych aplikacji jest skierowana jednak do zaawansowanych użytkowników,
oferuje obsługę wyłącznie poprzez interfejs graficzny, nie pozwalający na automatyzację procesu przetwarzania zdjęć, lub nie jest przystosowane do wykonania badań wydajnościowych i jakościowych. 
Celem niniejszej pracy jest stworzenie aplikacji desktopowej, która umożliwi przetwarzanie zdjęć głębokiego nieba z wykorzystaniem podstawowych technik obróbki obrazów astronomicznych 
umożliwiająca prostą obsługę dla użytkowników początkujących i średnio-zaawansowanych z możliwością obsługi poprzez linię komend i umożliwiającą przeprowadzanie badań wydajnościowych i jakościowych.
Dodatkowo aplikacja będzie umożliwiała prostą rozbudowę o dodatkowe techniki obróbki obrazów, utworzenie interfejsu graficznego aplikacji oraz dodanie nowych funkcjonalności badawczych w przyszłości.