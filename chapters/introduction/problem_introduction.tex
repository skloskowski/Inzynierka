\section{Opis problemu}

Przetwarzanie zdjęć głębokiego nieba stanowi istotny element amatorskiej astrofotografii. Obiekty głębokiego nieba znajdują się poza Układem Słonecznym, więc ich znaczna odległość powoduje, że 
ich pozorna jasność na nocnym niebie nie pozwala na dostrzeżenie ich gołym okiem, zwłaszcza w obszarach o dużym zanieczyszeniu świetlnym. Do pokonania tych ograniczeń amatorzy astrofotografii wykorzystują
fotografię z dłuższym czasem naświetlania, które pozwalają na zebranie większej ilości światła z odległych obiektów. Takie podejście niesie ze sobą szereg wymógów dotyczących sprzętu oraz odpowiedniej techniki.
Alternatywą do tego podejścia jest wykonanie serii krótkich ekspozycji, które są następnie łączone w jeden plik wynikowy - pozwala to na uproszczenie procesu zbierania danych. Możliwe jest łączenie
tych technik i wykonanie serii dłuższych ekspozycji, często w zakresach kilku sekund do kilku minut, które są następnie łączone w jeden plik wynikowy. Niezależnie od podejścia, zebrane dane wymagają
odpowiedniego przetworzenia.