\section{Wyniki eksperymentów}\label{chapter:results}

W tym rozdziale przedstawiono wyniki badań czasu działania programu w zależności od zastosowanej metody nakładania zdjęc oraz całkowitego czasu integracji zdjęć. 
Do interpretacji wyników należy pamiętać, że całkowity czas integracji nie jest porównywalny pomiędzy zestawami, ponieważ zestawy zdjęć wykorzysywały fotografie o różnym czasie naświetlania pojedynczego zdjęcia.
Zestaw 1499 wykorzystywał zdjęcia o czasie naświetlania 60 sekund w porównaniu do zestawu 244, który wykorzystywał zdjęcia o czasie naświetlania 120 sekund. Dodatkowy wpływ na wyniki czasu działania
Dodatkowy wpływ na wyniki czasu działania ma również wykorzystanie klatek kalibracyjnych do korekcji zdjęć przed nałożeniem w zestawie 1499. Badany został całkowity czas działania programu, który
obejmował czas wczytania zdjęć, wykonania kalibracji, rejestracji oraz nałożenia zdjęć. Do czasu działania programu nie wliczano jednak czasu potrzebnego na generację klatek kalibracyjnych.

Poniżej przedstawiono wyniki eksperymentów dla czasu działania programu dla zestawów 244 oraz 1499. Czas działania programu mierzono dla czterech metod nakładania zdjęć: średnia, mediana, Kappa-Sigma 
Clipping oraz Auto Adaptive Weighted Average. Wskazano różnice w czasie działania programu w zależności od całkowitego czasu integracji zdjęć. 
Dla każdej kombinacji czasu integracji oraz badanego algorytmu wykonano pięć iteracji programu obliczono następnie średni czas działania oraz błąd. Jako błąd pomiaru przyjęto jedno odchylenie standardowe. 
Przed obliczeniem ostatecznych wyników zdecydowano się na inwalidację błędów grubych, które zdefiniowano na czasy odbiegające znacznie od średniego czasu działania dla badanej kombinacji.
Jedyny taki błąd gruby zaobserwowano dla pojedynczego badania kombinacji zestawu 244 metodą Auto Adaptive Weighted Average przy czasie integracji 60 minut. Czas błędu grubego odstawał o ponad 50\% od średniej
wartości czasu działania dla tej kombinacji - po inwalidacji tego błędu grubego odchylenie standardowe pomiarów wyniosło poniżej 1 sekundy.
Na wykresach \ref{fig:zestaw244_time_chart} oraz \ref{fig:zestaw1499_time_chart} przedstawiono graficzne porównanie średnich czasów działannia programu w zależności do zastosowanej metody nakładania zdjęć 
oraz całkowitego czasu integracji odpowiednio dla zestawu 244 oraz zestawu 1499. Dokładne wartości średnich oraz błędów przedstawiono odpowiednio w tabelach \ref{tab:zestaw244_time_table} 
oraz \ref{tab:zestaw1499_time_table} dla zestawu 244 oraz zestawu 1499.

\newpage

\begin{figure}[!ht]
    \centering
    \includegraphics[width=0.8\linewidth]{images/zestaw244_time_chart.png}
    \caption{Porównanie średnich czasów nakładania zdjęć i odchylenia standardowego pomiarów w zależności od metody nakładania zdjęć i czasu integracji dla zestawów 244}
    \label{fig:zestaw244_time_chart}
\end{figure}

\begin{table}[!ht]
\centering
\begin{tabular}{llrrrrrr}
\toprule
Metoda &  & \multicolumn{6}{c}{Czas integracji [min]} \\
\cmidrule(lr){3-8}
 &  & 10 & 20 & 30 & 40 & 50 & 60 \\
\midrule
\multirow{2}{*}{Średnia} 
 & $\overline{t}$ [s] & 34.93 & 68.94 & 106.03 & 146.79 & 188.19 & 221.05 \\
 & $\sigma$ [s]        & 0.14  & 0.18  & 0.27   & 0.83   & 0.76   & 0.94   \\
\midrule
\multirow{2}{*}{Mediana} 
 & $\overline{t}$ [s] & 37.81 & 77.30 & 121.38 & 170.48 & 222.07 & 267.22 \\
 & $\sigma$ [s]       & 0.87  & 0.15  & 0.63   & 0.75   & 1.24   & 1.73   \\
\midrule
\multirow{2}{*}{Kappa–Sigma Clip.} 
 & $\overline{t}$ [s] & 69.82 & 137.88 & 206.07 & 273.56 & 349.24 & 410.79 \\
 & $\sigma$ [s]       & 0.27  & 7.82   & 1.06   & 0.33   & 0.69   & 0.77   \\
\midrule
\multirow{2}{*}{AA Weighted Avg.} 
 & $\overline{t}$ [s] & 80.13 & 152.66 & 225.90 & 307.11 & 388.05 & 459.12 \\
 & $\sigma$ [s]       & 0.09  & 1.34   & 0.22   & 0.89   & 0.61   & 0.92   \\
\bottomrule
\end{tabular}
\caption{Porównanie średnich czasów nakładania zdjęć w zależności od metody nakładania zdjęć i czasu integracji dla zestawów 244.}\
\label{tab:zestaw244_time_table}
\end{table}

\newpage

\begin{figure}[!ht]
    \centering
    \includegraphics[width=0.8\linewidth]{images/zestaw1499_time_chart.png}
    \caption{Porównanie średnich czasów nakładania zdjęć i odchylenia standardowego pomiarów w zależności od metody nakładania zdjęć i czasu integracji dla zestawów 1499}
    \label{fig:zestaw1499_time_chart}
\end{figure}

\begin{table}[!ht]
\centering
\begin{tabular}{llrrrrrr}
\toprule
Metoda &  & \multicolumn{6}{c}{Czas integracji [min]} \\
\cmidrule(lr){3-8}
 &  & 10 & 20 & 30 & 40 & 50 & 60 \\
\midrule
\multirow{2}{*}{Średnia} 
 & $\overline{t}$ [s] & 72.95 & 147.10 & 222.44 & 296.02 & 376.46 & 537.83 \\
 & $\sigma$ [s]       & 0.30  & 0.40  & 0.49   & 2.55   & 0.78   & 7.42   \\
\midrule
\multirow{2}{*}{Mediana} 
 & $\overline{t}$ [s] & 81.47 & 171.42 & 268.77 & 390.84 & 557.23 & 869.26 \\
 & $\sigma$ [s]       & 0.41  & 0.40  & 0.43   & 14.39  & 11.88  & 23.37  \\
\midrule
\multirow{2}{*}{Kappa–Sigma Clip.} 
 & $\overline{t}$ [s] & 143.72 & 272.69 & 405.22 & 511.84 & 688.99 & 1018.62 \\
 & $\sigma$ [s]       & 13.38  & 2.08   & 1.60   & 2.12   & 6.46   & 30.15  \\
\midrule
\multirow{2}{*}{AA Weighted Avg.} 
 & $\overline{t}$ [s] & 155.23 & 307.36 & 463.54 & 605.03 & 765.73 & 980.34 \\
 & $\sigma$ [s]       & 0.38   & 0.77   & 2.96   & 0.84   & 2.29   & 43.84  \\
\bottomrule
\end{tabular}
\caption{Porównanie średnich czasów nakładania zdjęć w zależności od metody nakładania zdjęć i czasu integracji dla zestawów 1499.}
\label{tab:zestaw1499_time_table}
\end{table}

\newpage

Poniżej przedstawiono wyniki eksperymentów dla FWHM zdjęć wynikowych dla zestawów 244 oraz 1499. FWHM zdjęć wynikowych mierzono dla czterech metod nakładania zdjęć: średnia, mediana, Kappa-Sigma 
Clipping oraz Auto Adaptive Weighted Average. Wskazano różnice w FWHM zdjęć w zależności od całkowitego czasu integracji zdjęć. Wyniki obliczenia FWHM są stałe dla pojedynczego zdjęcia i zależą tylko od
określonego okna pomiarowego oraz obliczonego centrum gwiazdy. Zdecydowano się więc na przedstawienie wyników FWHM jako pojedycznego pomiaru. Algorytm wykrywania gwiazd nie jest w pełni skuteczny, dlatego
zdecydowano się na jednokrotne ustalenie pozycji gwiazd oraz ich centrum dla podstawie zdjęcia referencyjnego, którym było zdjęcie wynikowe uzyskane metodą Auto Adaptive Weighted Average przy 
maksymalnym badanym czasie integracji. Badanie FWHM przeprowadzone w ten sposób ma jednak możliwość wprowadzenia błędów spowodowanych innym położeniem centrum gwiazdy na zdjęciach wynikowych uzyskanych 
przez różne algorytmy oraz różne czasy integracji. Podany sposób badań jest jednak niezbędny do eliminacji wpływu detekcji gwiazd na wyniki FWHM. Wyniki FWHM przedstawiono dla trzech różnych rozmiarów okna
pomiarowego: 11, 15 oraz 19 pikseli. Na wykresach \ref{fig:zestaw244_fwhm_11}, \ref{fig:zestaw244_fwhm_15} oraz \ref{fig:zestaw244_fwhm_19} przedstawiono graficzne porównanie wartości FWHM w zależności 
do zastosowanej metody nakładania zdjęć oraz całkowitego czasu integracji odpowiednio dla zestawu 244. Wyniki dla zestawu 1499 przedstawiono na wykresach \ref{fig:zestaw1499_fwhm_11}, \ref{fig:zestaw1499_fwhm_15}
 oraz \ref{fig:zestaw1499_fwhm_19}. Dokładne wartości wyników FWHM znajdują się odpowiednio w tabelach \ref{tab:zestaw_244_fwhm_table} oraz \ref{tab:zestaw_1499_fwhm_table} dla zestawu 244 oraz zestawu 1499.

\begin{figure}[!ht]
    \centering
    \includegraphics[width=0.8\linewidth]{images/zestaw244_fwhm_11.png}
    \caption{Porównanie FWHM zestawów 244 w zależności od metody nakładania zdjęć i czasu integracji dla okna 11 pikseli.}
    \label{fig:zestaw244_fwhm_11}
\end{figure}

\begin{figure}[!ht]
    \centering
    \includegraphics[width=0.8\linewidth]{images/zestaw244_fwhm_15.png}
    \caption{Porównanie FWHM zestawów 244 w zależności od metody nakładania zdjęć i czasu integracji dla okna 15 pikseli.}
    \label{fig:zestaw244_fwhm_15}
\end{figure}

\newpage

\begin{figure}[!ht]
    \centering
    \includegraphics[width=0.8\linewidth]{images/zestaw244_fwhm_19.png}
    \caption{Porównanie FWHM zestawów 244 w zależności od metody nakładania zdjęć i czasu integracji dla okna 19 pikseli.}
    \label{fig:zestaw244_fwhm_19}
\end{figure}

\begin{table}[!ht]
\centering
\begin{tabular}{llrrrrrr}
\toprule
Metoda &  & \multicolumn{6}{c}{Czas integracji [min]} \\
\cmidrule(lr){3-8}
 & & 10 & 20 & 30 & 40 & 50 & 60 \\
\midrule
\multirow{3}{*}{Średnia}
 & 11~px [px] & 4.475 & 4.512 & 4.523 & 4.539 & 4.560 & 4.568 \\
 & 15~px [px] & 4.982 & 4.991 & 5.017 & 5.013 & 5.023 & 5.039 \\
 & 19~px [px] & 5.508 & 5.509 & 5.483 & 5.490 & 5.494 & 5.506 \\
\midrule
\multirow{3}{*}{Mediana}
 & 11~px [px] & 4.479 & 4.503 & 4.486 & 4.514 & 4.532 & 4.549 \\
 & 15~px [px] & 4.968 & 4.998 & 4.948 & 4.963 & 4.988 & 4.987 \\
 & 19~px [px] & 5.474 & 5.497 & 5.449 & 5.475 & 5.455 & 5.491 \\
\midrule
\multirow{3}{*}{Kappa–Sigma Clip.}
 & 11~px [px] & 4.470 & 4.492 & 4.498 & 4.531 & 4.525 & 4.541 \\
 & 15~px [px] & 4.977 & 4.976 & 4.969 & 4.944 & 4.955 & 4.974 \\
 & 19~px [px] & 5.500 & 5.472 & 5.453 & 5.465 & 5.449 & 5.460 \\
\midrule
\multirow{3}{*}{AA Weighted Avg.}
 & 11~px [px] & 4.489 & 4.495 & 4.521 & 4.526 & 4.528 & 4.561 \\
 & 15~px [px] & 4.966 & 4.997 & 4.979 & 4.968 & 4.991 & 5.010 \\
 & 19~px [px] & 5.455 & 5.478 & 5.457 & 5.462 & 5.477 & 5.478 \\
\bottomrule
\end{tabular}
\caption{Porównanie wartości FWHM w zależności od metody nakładania zdjęć i czasu integracji dla zestawów 244.}
\label{tab:zestaw_244_fwhm_table}
\end{table}

\newpage

\begin{figure}[!ht]
    \centering
    \includegraphics[width=0.8\linewidth]{images/zestaw1499_fwhm_11.png}
    \caption{Porównanie FWHM zestawów 1499 w zależności od metody nakładania zdjęć i czasu integracji dla okna 11 pikseli.}
    \label{fig:zestaw1499_fwhm_11}
\end{figure}

\begin{figure}[!ht]
    \centering
    \includegraphics[width=0.8\linewidth]{images/zestaw1499_fwhm_15.png}
    \caption{Porównanie FWHM zestawów 1499 w zależności od metody nakładania zdjęć i czasu integracji dla okna 15 pikseli.}
    \label{fig:zestaw1499_fwhm_15}
\end{figure}

\newpage

\begin{figure}[!ht]
    \centering
    \includegraphics[width=0.8\linewidth]{images/zestaw1499_fwhm_19.png}
    \caption{Porównanie FWHM zestawów 1499 w zależności od metody nakładania zdjęć i czasu integracji dla okna 19 pikseli.}
    \label{fig:zestaw1499_fwhm_19}
\end{figure}

\begin{table}[!ht]
\centering
\begin{tabular}{llrrrrrr}
\toprule
Metoda &  & \multicolumn{6}{c}{Czas integracji [min]} \\
\cmidrule(lr){3-8}
 & & 10 & 20 & 30 & 40 & 50 & 60 \\
\midrule
\multirow{3}{*}{Średnia}
 & 11~px [px] & 4.511 & 4.517 & 4.564 & 4.594 & 4.658 & 4.679 \\
 & 15~px [px] & 5.099 & 5.128 & 5.134 & 5.157 & 5.169 & 5.173 \\
 & 19~px [px] & 5.651 & 5.655 & 5.615 & 5.639 & 5.619 & 5.633 \\
\midrule
\multirow{3}{*}{Mediana}
 & 11~px [px] & 4.509 & 4.517 & 4.532 & 4.595 & 4.639 & 4.645 \\
 & 15~px [px] & 5.091 & 5.070 & 5.145 & 5.162 & 5.148 & 5.147 \\
 & 19~px [px] & 5.669 & 5.639 & 5.633 & 5.607 & 5.612 & 5.608 \\
\midrule
\multirow{3}{*}{Kappa–Sigma Clip.}
 & 11~px [px] & 4.492 & 4.503 & 4.530 & 4.570 & 4.616 & 4.645 \\
 & 15~px [px] & 5.090 & 5.101 & 5.121 & 5.119 & 5.133 & 5.161 \\
 & 19~px [px] & 5.650 & 5.638 & 5.583 & 5.610 & 5.613 & 5.607 \\
\midrule
\multirow{3}{*}{AA Weighted Avg.}
 & 11~px [px] & 4.514 & 4.505 & 4.555 & 4.594 & 4.618 & 4.660 \\
 & 15~px [px] & 5.087 & 5.082 & 5.133 & 5.145 & 5.151 & 5.158 \\
 & 19~px [px] & 5.694 & 5.628 & 5.611 & 5.602 & 5.617 & 5.594 \\
\bottomrule
\end{tabular}
\caption{Porównanie wartości FWHM w zależności od metody nakładania zdjęć i czasu integracji dla zestawów 1499.}
\label{tab:zestaw_1499_fwhm_table}
\end{table}

\newpage

Poniżej przedstawino wizualne porównanie tych samych wycinków zdjęć wynikowych przedstawiających ten sam obiekt dla różnych algorytmów nakładania zdjęć. Każde ze zdjęć wynikowych zostało uzyskane poprzez
nałożenie zdjęć z całkowitym czasem integracji 60 minut. Do wykonania zdjęć wynikowych obiektu 244 nie użyto klatek kalibracyjnych, natomiast do wykonania zdjęć wynikowych obiektu 1499
użyto klatki kalibracyjne. Każda z fotografii wynikowych została zmodyfikowano w ten sam sposób w celu poprawy widoczności szczegółów na zdjęciach - zastosowano krzywe oraz zwiększono nasycenie kolorów.
Do edycji zdjęć użyto darmowy program GNU Image Manipulation Program (GIMP) możliwy do pobrania z oficjalnej strony producenta \cite{gimp}.
Dla zestawu 244 obiektem fotografowanym był obiekt NGC 244 - galaktyka spiralna w gwiazdozbiorze Andromedy znana jako galaktyka Andromedy. 
Porównanie wizualne zdjęć zestawu 244 przedstawiono na rysunku \ref{fig:zestaw244_60_comp}.
Dla zestawu 1499 obiektem fotografowanym był obiekt NGC 1499 - mgławica emisyjna w gwiazdozbiorze Perseusza znana jako mgławica Kalifornia. 
Porównanie wizualne zdjęć zestawu 1499 przedstawiono na rysunku \ref{fig:zestaw1499_60_comp}.
Ze względu na różne warunki obserwacyjne, jasność nieba w zenicie oraz różny czas naświetlania pojedynczego zdjęcia, fotografie wynikowe nie powinny zostać porównywane pomiędzy zestawami.

\begin{figure}[!ht]
    \centering
    \includegraphics[width=0.55\linewidth]{images/zestaw244_60_comp.png}
    \caption{Porównanie wizualne obiektu z zestawu 244 o całkowitym czasie integracji 60 min. dla różnych algorytmów.}
    \label{fig:zestaw244_60_comp}
\end{figure}

\begin{figure}[!ht]
    \centering
    \includegraphics[width=0.55\linewidth]{images/zestaw1499_60_comp.png}
    \caption{Porównanie wizualne obiektu z zestawu 1499 o całkowitym czasie integracji 60 min. dla różnych algorytmów.}
    \label{fig:zestaw1499_60_comp}
\end{figure}

\newpage
\clearpage

Poniżej przedstawiono wizualne porównanie tych samych wycinków zdjęć wynikowych dla różnych czasów integracji zdjęć. Obrazy wynikowe zostały wykonane korzystając z algorytmu Kappa-Sigma Clipping.
Każde ze zdjęć wynikowych zostało uzyskane poprzez nałożenie zdjęć z różnym całkowitym czasem integracji opisanym na rysunku.  Każda z fotografii wynikowych została zmodyfikowano w ten sam sposób
w celu poprawy widoczności szczegółów na zdjęciach - zastosowano krzywe oraz zwiększono nasycenie kolorów. Do wykonania zdjęć wynikowych obiektu 244 nie użyto klatek kalibracyjnych,
natomiast do wykonania zdjęć wynikowych obiektu 1499 użyto klatki kalibracyjne. Ze względu na różne warunki obserwacyjne, jasność nieba w zenicie oraz różny czas naświetlania pojedynczego zdjęcia,
fotografie wynikowe nie powinny zostać porównywane pomiędzy zestawami.

\begin{figure}[!ht]
    \centering
    \includegraphics[width=0.8\linewidth]{images/zestaw244_noise_kappa_comp.png}
    \caption{Porównanie wizualne szumu dla różnego czasu integracji algorytmu Kappa-Sigma Clipping zestawów 244.}
    \label{fig:zestaw244_noise_kappa_comp}
\end{figure}

\newpage

\begin{figure}[!ht]
    \centering
    \includegraphics[width=0.8\linewidth]{images/zestaw1499_noise_kappa_comp.png}
    \caption{Porównanie wizualne szumu dla różnego czasu integracji algorytmu Kappa-Sigma Clipping zestawów 1499.}
    \label{fig:zestaw1499_noise_kappa_comp}
\end{figure}

\newpage

\clearpage
