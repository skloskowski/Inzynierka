\section{Analiza wyników}\label{chapter:result_comparison}

W niniejszym rozdziale dokonano analizy wyników eksperymentów zebranych w rozdziale \ref{chapter:results}. Skupiono się na dwóch głównych aspektach: czasie wykonania poszczególnych metod
nakładania zdjęć oraz jakości obrazu mierzonej za pomocą FWHM. Określenie czy przedstawione dane można interpretować za liniowe bądź wykładnicze określono korzystając z programu Microsoft Excel
na podstawie funkcji dopasowania krzywej trendu do danych eksperymentalnych. Określenie dopasowania krzywej trendu do danych eksperymentalnych dokonano na podstawie współczynnika determinacji $R^2$
- linia trendu z najwyższą wartością współczynnika była uznawana za najlepszą reprezentację krzywej do danych. Współczynnik determinacji $R^2$ definiowany jest wzorami \ref{determination_coefficient}
na podstawie \cite{coefficientofdetermination}.

\begin{equation}\label{determination_coefficient}
    R^2 = 1 - \frac{SS_{res}}{SS_{tot}} \qquad
    SS_{res} = \sum_{i=1}^{n}(y_i - f_i)^2 \quad
    SS_{tot} = \sum_{i=1}^{n}(y_i - \bar{y})^2
\end{equation}

gdzie,

\begin{tabular}{l}
$y_i$ - wartość obserwowana \\
$f_i$ - wartość przewidywana przez model \\
$\bar{y}$ - średnia wartość obserwowana \\
\end{tabular}

\vspace{0.5cm}

Analiza wyników eksperymentów dla zestawu 244 zawartych w tabeli \ref{tab:zestaw244_time_table} oraz na wykresie \ref{fig:zestaw244_time_chart} pokazuje wyraźne różnice w czasie wykonania pomiędzy
metodami prostymi: średnia oraz mediana, a metodami złożonymi: Kappa-Sigma Clipping oraz Auto Adaptive Weighted Average. Wyznaczona krzywa wskazuje na to, że do 30 zdjęć wszystkie z badanych algorytmów
mają liniowy czas wykonania. Wzrost całkowitego czasu integracji powoduje większy wzrost czasu działania programu dla algorytmu Auto Adaptive Weighted Average w porównaniu do Kappa-Sigma Clipping, który wynosi od 
około 15\% różnicy dla pięciu łączonych zdjęć (oznaczonych jako całkowity czas integracji równy 10 minut) do prawie 18\% różnicy dla trzydziestu łączonych zdjęć. Różnice w czasie działania programu
pomiędzy algorytmami średniej i mediany są znacznie większe wynosząc od 8\% dla pięciu łączonych zdjęć do ponad 20\% dla trzydziestu łączonych zdjęć. Największą odpornością na wzrost czasu działania programu
względem całkowitego czasu integracji charakteryzuje się algorytm Auto Adaptive Weighted Average, którego czas działania programu wzrósł o 573\% pomiędzy pięcioma a trzydziestoma łączonymi zdjęciami, 
najmniejszą odporność wykazuje algorytm mediany, którego czas działania wzrósł o ponad 706\% w tym samym zakresie.

Analiza wyników eksperymentów dla zestawu 1499 zawartych w tabeli \ref{tab:zestaw1499_time_table} oraz na wykresie \ref{fig:zestaw1499_time_chart} pokazuje na nagły wzrost czasu działania programu
powyżej sześćdziesięciu łączonych zdjęć dla wszystkich badanych algorytmów. Największym wzrostem czasu działania programu od pięćdziesięciu do sześćdziesięciu łączonych zdjęć charakteryzują się algorytmy 
Kappa-Sigma Clipping oraz mediany, dla których średni czas działania programu wzrósł o odpowiednio 48\% oraz 56\%. Najmniejszym wzrostem czasu działania w tym zakresie charakteryzuje się algorytm 
Auto Adaptive Weighted Average, którego średni czas działania programu wzrósł o 28\%. Algorytm mediany wykazuje najmniejszą odporność na wzrost czasu działania najprawdopodobniej ze względu na
konieczność sortowania wartości pikseli w obrazach wejściowych. W celu zmniejszenia złożoności obliczeniowej algorytmu mediany, należy rozważyć wykorzystanie algorytmu Bluma-Floyda-Pratta-Rivesta-Tarjana. 
Algorytm ten pozwala na znalezienie mediany w czasie liniowym \cite{medianofmedians}.
Dla metody Kappa-Sigma Clipping nie zastosowano optymalizacji pamięciowych, tak jak w algorytmie Auto Adaptive Weighted Average, co może być powodem nagłego wzrostu czasu działania.
W celu dostarczenia dokładniejszej analizy czasów działania należy wykonać badania zużycia pamięci przez poszczególne algorytmy oraz badania dla większej ilości danych.

Analiza eksperymentów zależności FWHM dla całkowitego czasu integracji bazowana na wynikach przedstawionych na wykresach \ref{fig:zestaw244_fwhm_11} - \ref{fig:zestaw1499_fwhm_19} oraz w tabelach
\ref{tab:zestaw_244_fwhm_table} oraz \ref{tab:zestaw_1499_fwhm_table}, rozmiaru okna pomiarowego oraz metody nakładania zdjęć dla obu badanych zestawów danych nie wykazała jednoznacznie przewagi żadnej
z metod nakładania zdjęć na ostrość obrazu wynikowego dla wszystkich badanych rozmiarów okna. Drobne różnice są obserwowane lokalnie - algorytm Kappa-Sigma Clipping reprezentauje najniższe wartości FWHM
dla największych czasów integracji dla wszystkich okien pomiarowych w zestawie 244 oraz dla okna 11 pikseli w zestawie 1499, wielkość tych różnic jest jednak niewielka. Algorytm średniej dla więkoszści
badanych przypadków charakteryzuje się najwyższymi wartościami FWHM. Wynika to prawdopodobnie z braku eliminacji wartości odstających, które mogą powodować rozmycie obrazu końcowego.
Brak idealnego dopasowania transformacji afinicznej pomiędzy poszczególnymi zdjęciami w zestawie może również powodować rozmycie obrazu końcowego, w celu dokładniejszej weryfikacji wyników tego zjawiska
należy zbadań wpływ dokładności dopasowania transformacji na wartość FWHM obrazów wynikowych - obecnie stosowana tolerancja jednego piksela może mieć znaczący wpływ na końcowe wartości FWHM zważając na fakt,
że różnice pomiędzy algorytmami w badanych przypadkach są o rząd wielkości mniejsze. 
Dla najmniejszego okna pomiarowego wartości FWHM mają tendecję do niewielkiego wzrostu wraz ze wzrostem całkowitego czasu integracji, natomiast dla okien 15 i 19 pikseli zależność jest mniej spójna
i niemożliwe jest wyciągnięcie jednoznacznych wniosków. Zależności opisane powyżej sugerują, że przyrost SNR związany z dłuższym czasem integracji nie przekłada się bezpośrednio na istotne
zmniejszenie FWHM. W celu dokładniejszej analizy wyników FWHM należy przeprowadzić dodatkowe badania zależności FWHM danych wejściowcych zmiany wartości FWHM obrazów wynikowych. Zmienne wartości widzenia,
temperatury i ostrości zdjęć w ciągu nocy obserwacyjnej mogą mieć wpływ na ostateczne wartości FWHM obrazów wynikowych - te zmiany mogą mieć różny wpływ w zależności od badanego algorytmu.

Porównania wizualne edytowanych plików wynikowych przedstawionych na rysunkach \ref{fig:zestaw244_60_comp} oraz \ref{fig:zestaw1499_60_comp} pozwalają na subiektywną ocenę jakości obrazu 
uzyskanego różnymi algorytmami pod względem poziomu szumu oraz zachowania ostrości szczegołów. Algorytmy Kappa-Sigma Clipping oraz Auto Adaptive Weighted Average wykazują nieznaczną przewagę 
w poziomie tła widoczną głównie w obszarach pozbawionych gwiazd oraz w zachowaniu struktur o niskim kontraście - różnice są jednak subtelne. 
Na porównaniach wizualnych zawartych na rysunkach \ref{fig:zestaw244_noise_kappa_comp} oraz \ref{fig:zestaw1499_noise_kappa_comp} skupiono się na ocenie wpływu całkowitego czasu integracji dla wybranego algorytmu.
Ze względu na wyniki wizualne oraz FWHM zdecydowano się na wykorzystanie algorytmu Kappa-Sigma Clipping jako przykład - tendencje spadkowe pozostają jednak podobne dla wszystkich algorytmów. 
Porównania te pokazują spadek poziomu szumu wraz ze wzrostem całkowitego czasu integracji zgodnie z oczekiwaniami. Efekt jest bardziej widoczny w zestawie 1499 - jest to możliwe ze względu na 
zastosowaną kalibrację zdjęć wejściowych w przeciwieństwie do zestawu 244, gdzie nie wykorzytano zdjęć kalibracyjnych. Alternatywnie, większy spadek w zesatwi 1499 można tłumaczyć wyższym 
poziomem zanieczyszeia świetlnego. W takich warunkach wydłużenie czasu integracji wykazuje wyraźniejszą poprawę stosunku sygnału do szumu, ponieważ SNR wzrasta wraz z pierwiastkiem czasu.
W celu obiektywnej oceny jakości obrazu należy przeprowadzić analizę algorytymiczną poziomu sygnału do szumu. Podczas tej pracy nie udało się przeprowadzić analizy SNR ze względu na wymagany czas 
opracowania odpowiednich narzędzi do pomiaru SNR w obrazach astronomicznych.
