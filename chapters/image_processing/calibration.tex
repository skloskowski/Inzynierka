\section{Kalibracja}

Jednym z niezbędnych elementów przetwarzania zdjęć jest zastosowanie klatek kalibracyjnych, które odpowiadają za korekcję różnych niedoskonałości teleskopu i matrycy. 
Poniżej opisano najczęściej używane klatki kalibracyjne. Dokładny opis założeń korzystania z nich opisano w książce \cite{Howell2006}. Dotyczy ona głównie astrofotografii na aparatach CCD, lecz podstawowe założenia pozostają 
niezmienne dla nowocześniejszych matryc CMOS. 

\begin{itemize}
    \item \textbf{Klatki Bias}

    Klatki bias, nazywane też klatkami offset, wykonywane są przy całkowicie zakrytej matrycy i czasie ekspozycji możliwie najkrótszym dla wykorzystywanej matrycy. Należy pamiętać o potrzebie wykonania tych klatek dla tych samych ustawień ISO 
    lub Gain (w zależności od zastosowanego rodzaju aparatu) jak główne fotografie obiektu. Stosuje się je w celu usunięcia stałego szumu generowanego przez elektronikę w trakcie odczytu informacji z matrycy. Klatki bias stanowią minimalny poziom 
    szumu zawartego w każdej wykonanej fotografii.
    
    \item \textbf{Klatki Dark}

    Klatki dark wykonuje się przy zakrytej matrycy, w tej samej temperaturze, takim samym czasem ekspozycji i ustawieniami ISO/Gain jak klatki obiektu. Ich głównym zadaniem jest pozbycie się prądu ciemnego (\textit{ang. dark current}), 
    redukcja zjawiska ,,amp glow'', pozbycie się gorących pikseli i wpływu promieniowania kosmicznego.
    
    \item \textbf{Klatki Flat}

    Klatki flat służą do usunięcia nierównomiernego oświetlenia matrycy (zjawisko winiety) i różnic czułości poszczególnych pikseli. Klatki flat przydają się dodatkowo przy zniwelowaniu artefaktów spowodowanych zanieczyszczeniami na matrycy oraz teleskopie. 
    Wykonywane są przy równomiernym oświetleniu całego pola widzenia - kierując teleskop lub obiektyw na jasne jednolite tło. Podczas kalibracji warto zwrócić uwagę na potrzebę odjęcia bias przed użyciem klatek flat na głównych fotografiach obiektu. 
\end{itemize}

    W celu osiągnięcia wymaganych efektów wykonuje się wiele klatek kalibracyjnych każdego rodzaju, a następnie przetwarza 
    się je obliczając średnią wartość na każdym pikselu. Ostateczny teoretyczny opis nałożenia klatek można opisać tak jak 
    w równaniu \eqref{eq:calibration}

\begin{equation}\label{eq:calibration}
    I_{\text{calibrated (x,y)}} = \frac{I_{\text{light (x,y)}} - I_{\text{dark (x,y)}}}{I_{\text{flat (x,y)}} - I_{\text{bias (x,y)}}}
\end{equation}

gdzie: \\

\begin{tabular}{l}

    \( I_{\text{light (x,y)}} \) - wartości pikseli (x,y) w surowej klatce obiektu \\
    \( I_{\text{dark (x,y)}} \) - wartości pikseli (x,y) w uśrednionej klatce dark \\
    \( I_{\text{bias (x,y)}} \) - wartości pikseli (x,y) w uśrednionej klatce bias \\
    \( I_{\text{flat (x,y)}} \) - wartości pikseli (x,y) w uśrednionej klatce flat

\end{tabular}
