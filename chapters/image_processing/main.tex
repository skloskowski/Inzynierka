\chapter{Przetwarzanie zdjęć}\label{chapter:image_processing}

W celu wytworzenia obrazu lepszej jakości oraz z niskim poziomem szumu fotografie obiektów głębokiego należy odpowiednio przetworzyć. Przetwarzanie zdjęć można podzielić na trzy główne etapy: kalibrację, rejestrację oraz stacking.

Prawidłowe opisanie i porównanie źródeł szumu na różnych urządzeniach oraz przy różnych warunkach wymaga ustalenia odpowiedniej jednostki. Każdy foton światła padający na piksel detektora w zależności od długości światła powoduje pewną szansę na wygenerowanie elektronu - prawdopodobieństwo uwolnienia się takiego elektronu nazywa się wydajnością kwantową (\textit{ang. Quantum Efficiency}). Standardowo ilość szumu na zdjęciu producenci określają za pomocą ilości elektronów wygenerowanych przy pobieraniu informacji z matrycy fotograficznej \cite{Howell2006}. Przydatne informacje dotyczące szumu generowanego na popularnych matrycach aparatów można znaleźć na stronie Photons to Photos \cite{photonstophotos}.

\section{Kalibracja}

Jednym z niezbędnych elementów przetwarzania zdjęć jest zastosowanie klatek kalibracyjnych, które odpowiadają za korekcję różnych niedoskonałości teleskopu i matrycy. 
Poniżej opisano najczęściej używane klatki kalibracyjne. Dokładny opis założeń korzystania z nich opisano w książce \cite{Howell2006}. Dotyczy ona głównie astrofotografii na aparatach CCD, lecz podstawowe założenia pozostają 
niezmienne dla nowocześniejszych matryc CMOS. 

\begin{itemize}
    \item \textbf{Klatki Bias}

    Klatki bias, nazywane też klatkami offset, wykonywane są przy całkowicie zakrytej matrycy i czasie ekspozycji możliwie najkrótszym dla wykorzystywanej matrycy. Należy pamiętać o potrzebie wykonania tych klatek dla tych samych ustawień ISO 
    lub Gain (w zależności od zastosowanego rodzaju aparatu) jak główne fotografie obiektu. Stosuje się je w celu usunięcia stałego szumu generowanego przez elektronikę w trakcie odczytu informacji z matrycy. Klatki bias stanowią minimalny poziom 
    szumu zawartego w każdej wykonanej fotografii.
    
    \item \textbf{Klatki Dark}

    Klatki dark wykonuje się przy zakrytej matrycy, w tej samej temperaturze, takim samym czasem ekspozycji i ustawieniami ISO/Gain jak klatki obiektu. Ich głównym zadaniem jest pozbycie się prądu ciemnego (\textit{ang. dark current}), 
    redukcja zjawiska ,,amp glow'', pozbycie się gorących pikseli i wpływu promieniowania kosmicznego.
    
    \item \textbf{Klatki Flat}

    Klatki flat służą do usunięcia nierównomiernego oświetlenia matrycy (zjawisko winiety) i różnic czułości poszczególnych pikseli. Klatki flat przydają się dodatkowo przy zniwelowaniu artefaktów spowodowanych zanieczyszczeniami na matrycy oraz teleskopie. 
    Wykonywane są przy równomiernym oświetleniu całego pola widzenia - kierując teleskop lub obiektyw na jasne jednolite tło. Podczas kalibracji warto zwrócić uwagę na potrzebę odjęcia bias przed użyciem klatek flat na głównych fotografiach obiektu. 
\end{itemize}

    W celu osiągnięcia wymaganych efektów wykonuje się wiele klatek kalibracyjnych każdego rodzaju, a następnie przetwarza 
    się je obliczając średnią wartość na każdym pikselu. Ostateczny teoretyczny opis nałożenia klatek można opisać tak jak 
    w równaniu \eqref{eq:calibration}

\begin{equation}\label{eq:calibration}
    I_{\text{calibrated (x,y)}} = \frac{I_{\text{light (x,y)}} - I_{\text{dark (x,y)}}}{I_{\text{flat (x,y)}} - I_{\text{bias (x,y)}}}
\end{equation}

gdzie: \\

\begin{tabular}{l}

    \( I_{\text{light (x,y)}} \) - wartości pikseli (x,y) w surowej klatce obiektu \\
    \( I_{\text{dark (x,y)}} \) - wartości pikseli (x,y) w uśrednionej klatce dark \\
    \( I_{\text{bias (x,y)}} \) - wartości pikseli (x,y) w uśrednionej klatce bias \\
    \( I_{\text{flat (x,y)}} \) - wartości pikseli (x,y) w uśrednionej klatce flat

\end{tabular}

\section{Rejestracja}

Rejestracja jest procesem przetwarzania zdjęć głębokiego nieba, która polega na odpowiedniej transformacji fotografii przed nakładaniem zdjęć tak, żeby przedstawiały one ten sam fragment nieba, niezależnie od przesunięć spowodowanych różnymi czynnikami 
mechanicznymi. Proces rejestracji można podzielić na dwa etapy. 

Pierwszym z nich będzie identyfikacja elementów, które pozostają niezmienne na obu fotografiach. Dla astrofotografii elementy te są proste do określenia i są to zazwyczaj gwiazdy widoczne w tle fotografowanego obiektu głębokiego nieba. Do wykrywania tych punktów 
można wykorzystać algorytmy, bazujące na lokalnych właściwościach obiektów. Algorytmy te wykrywają punkty o jednolitej jasności, różniące się od otoczenia poprzez analizę jasności, kształtu i rozmiaru obiektów. Oblicza się następnie punkt centralny 
wykrytych obiektów, żeby można było je stosować jako punkty odniesienia.

Następnym etapem będzie określenie jaką transformację trzeba dokonać, żeby surową klatkę nałożyć prawidłowo na fotografię referencyjną. 
Jednym z algorytmicznych sposobów rozwiązania tego problemu będzie określenie stosunków długości boków dla wszystkich zestawów trzech gwiazd, znajdujących się na obu zdjęciach. 
Należy wtedy znaleźć najbliżej pasujące stosunki trójkątów pomiędzy zdjęciem referencyjnym, a surową klatką i oszacowanie macierzy transformacji pomiędzy zdjęciami. 
Jest to metoda uproszczona, inspirowana na bardziej zaawansowanych rozwiązaniach zaproponowanych przez Lin i Xu w artykule \cite{rs15071921}.

Alternatywą dla tego podejścia jest astrometryczne rozwiązywanie obrazu (ang. \textit{Plate Solving}), polegające na dopasowaniu układu gwiazd z obrazu do katalogów gwiezdnych, w celu uzyskania dokładnych parametrów orientacji, skali oraz 
położenia pola widzenia obrazu, względem nocnego nieba \cite{doi:10.2514/6.2025-99706}.
\section{Stacking}

Stacking zdjęć jest procesem przetwarzania cyfrowych zdjęć polegającym na połączeniu wielu fotografii tego samego wycinka nocnego nieba lub wskazanego obiektu głębokiego nieba w celu osiągnięcia \
większego stosunku sygnału do zakłóceń (ang. \textit{Signal-to-Noise Ratio}, SNR) oraz ogólnej redukcji szumu na końcowym zdjęciu \cite{Hainaut2005}. Równania \eqref{eq:snr} i \eqref{eq:snr_stack} zostały 
zaczerpnięte z \cite{Hainaut2005}. Relację sygnału do szumu dla pojedynczej ekspozycji przedstawia równanie \eqref{eq:snr}:

\begin{equation}\label{eq:snr}
\text{SNR} = \frac{S}{\sqrt{S + \text{Sky} + \text{Dark} + N_{\mathrm{RON}}^2}}
\end{equation}

gdzie: \\

\begin{tabular}{l}
$S = s \cdot t = s \cdot N_\mathrm{DIT} \cdot \mathrm{DIT}$ – całkowity sygnał pochodzący od obiektów astronomicznych [\elec] \\
$\text{Sky}$ – szum tła nieba [\elec] \\
$\text{Dark}$ – szum prądu ciemnego [\elec] \\
$N_{\mathrm{RON}}$ – szum odczytu detektora [\elec] \\
$N_\mathrm{DIT}$ – liczba ekspozycji \\
$\mathrm{DIT}$ – czas pojedynczej ekspozycji [s] \\
$s$ – liczba elektronów zgromadzonych na sekundę od obiektów astronomicznych [\elec] \\ \\
\end{tabular}

Dla wielu połączonych ekspozycji SNR wzrasta w przybliżeniu jak pierwiastek liczby klatek, co pokazuje równanie \eqref{eq:snr_stack}, gdzie $N$ oznacza liczbę użytych klatek w procesie stackingu.

\begin{equation}\label{eq:snr_stack}
\text{SNR}_\text{stack} \approx \sqrt{N} \cdot \text{SNR}_\text{single}
\end{equation}

Programy do przetwarzania fotografii obiektów głębokiego nieba oferują wiele algorytmów do tego procesu. Przeanalizowano jednak cztery metody, które reprezentują różne podejścia do poprawy jakości obrazu.

\begin{itemize}
    \item \textbf{Średnia}
    
    Metoda średniej jest jedną z najczęściej stosowanych ze względu na prostotę implementacji oraz niskie wymagania sprzętowe, co stanowi istotną zaletę przy przetwarzaniu dużych zbiorów danych. Każdy piksel wynikowego obrazu jest obliczany jako średnia arytmetyczna wartości odpowiadających pikseli ze wszystkich klatek, jak przedstawiono w pseudokodzie \ref{lst:average}.

    \begin{lstlisting}[language=Python, caption={Pseudokod metody Średnia}, label={lst:average}]
for each pixel (x, y):
    sum = 0
        for each frame in frames:
            sum = sum + frame[x, y]
        output[x, y] = sum / number_of_frames
    \end{lstlisting}
    
    \item \textbf{Mediana}

    Metoda mediany, pokazana w psuedokodzie \ref{lst:median}, opiera się na wybraniu mediany z wartości pikseli w tym samym miejscu dla wszystkich klatek. Dzięki temu skutecznie usuwa anomalie takie jak ślady po satelitach czy pojedyncze piksele zakłóceń.

    \begin{lstlisting}[language=Python, caption={Pseudokod metody Mediana}, label={lst:median}]
for each pixel (x, y):
    values = []
    for each frame in frames:
        values.append(frame[x, y])
    sort(values)
    output[x, y] = median(values)
    \end{lstlisting}
    
    \item \textbf{Kappa-Sigma Clipping}

    Metoda Kappa-Sigma Clipping polega na iteracyjnym odrzucaniu wartości pikseli, które odbiegają od średniej o więcej niż $\kappa$-krotność odchylenia standardowego. Proces ten pozwala na ograniczenie 
    wpływów odchyleń takich jak samoloty oraz satelity kosmiczne zachowując dobry stosunek sygnału do szumu. Pseudokod tej metody przedstawiono w \ref{lst:kappasigma}. Wartości parametrów $\kappa$ oraz 
    maksymalnej liczby iteracji można dostosować w zależności od charakterystyki danych wejściowych - wartości stosowane w implementacji zostały opisane w rodziale \ref{chapter:code_structure}.

    \begin{lstlisting}[language=Python, caption={Pseudokod metody Kappa-Sigma Clipping}, label={lst:kappasigma}]
for each pixel (x, y):
    values = []
    for each frame in frames:
        values.append(frame[x, y])

    for iteration = 1 to max_iterations:
        mean = average(values)
        stddev = standard_deviation(values)
        inliers = []
        for each v in values:
            if abs(v - mean) <= kappa * stddev:
                inliers.append(v)
        if inliers = values or inliers is empty:
            break
        values = inliers
    \end{lstlisting}
    
    \item \textbf{Auto Adaptive Weighted Average}

    Metoda Auto Adaptive Weighted Average jest algorytmem iteracyjnym, która przypisuje każdemu pikselowi wagę zależną od odchylenia od bieżącej średniej. Wagi są wyznaczane za pomocą funkcji adaptacyjnej 
    kontrolowanej parametrami $\alpha$ oraz $\beta$. Należy jednak najpierw wybrać wartość początkową do dalszej iteracji, którą w tym przypadku jest mediana wartości pikseli.
    Metoda pozwala na większa swobodę w implementacji - funkcja adaptacyjna może być zależna nie tylko od wartości pikseli na fotografii ale również od 
    statystycznie obliczonej jakości zdjęcia czy innych wymaganych parametrów. Założenia algorytmu przedstawiono w kodzie \ref{lst:adaptive}. Wartości parametrów oraz funkcja ważenia zostały opisane
    w rozdziale \ref{chapter:code_structure}.

    \begin{lstlisting}[language=Python, caption={Pseudokod metody Auto Adaptive Weighted Average}, label={lst:adaptive}]
for each pixel (x, y):
    values = []
    for each frame in frames:
        values.append(frame[x, y])
    mu = median(values)
    
    for iteration = 1 to max_iterations:
        weights = []
        for i = 1 to length(values):
            w = adaptive_weight(values[i], mu, alpha, beta)
            weights.append(w)

        total_weight = sum(weights)
        for i = 1 to length(weights):
            weights[i] /= total_weight

        mu = 0
        for i = 1 to length(values):
            mu += weights[i] * values[i]

    output[x, y] = mu
\end{lstlisting}
    
\end{itemize}
