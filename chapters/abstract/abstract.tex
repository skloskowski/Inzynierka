\section*{Streszczenie}\label{chapter:abstract}

Celem niniejszej pracy było opracowanie i implementacja aplikacji desktopowej do przetwarzania obrazów głębokiego nieba z wykorzystaniem podstawowych technik obróbki zdjęć astronomicznych.
Rozwiązanie skierowane jest do użytkowników początkujących i rozpoczynających pracę z astrofotografią, oraz osób ceniących możliwość automatyzacji procesu przetwarzania zdjęć, a także możliwość wykonywania badań wydajnościowych i jakościowych.
W ramach pracy zaimplementowano aplikację desktopową korzystając z języka C++, bibliotek OpenCV oraz LibRaw z istotnymi funkcjonalnościami przetwarzania zdjęć głębokiego nieba: kalibracją, rejestracją oraz nakładaniem zdjęć (stacking). Dbano również o stworzenie 
modularnej architektury umożliwiającą łatwą rozbudowę aplikacji o dodatkowe techniki obróbki i funkcjonalności badawcze w przyszłości. Przeprowadzone zostały testy wydajnościowe aplikacji oraz jakościowe plików wynikowych, na podstawie rzeczywistych danych astronomicznych.
Analiza otrzymanych wyników potwierdziła poprawność działania programu oraz przydatność aplikacji w środowisku amatorskiej astrofotografii. Przedstawione zostały również propozycje zastosowania zaimplementowanych technik,
kierunki dalszego rozwoju aplikacji i przeprowadzanych badań w przyszłości.

\section*{Abstract}

The aim of this thesis was to design and implement a desktop application for deep-sky image processing using basic astronomical image processing techniques, aimed at amateur astrophotographers. The solution is intended for beginners starting their 
work with astrophotography and people valuing automation of the image processing workflow, as well as the possibility of performing performance and quality benchmarks. As part of the work a desktop application was implemented using C++ and 
OpenCV and LibRaw libraries with essential deep-sky image processing functionalities: calibration, registration and stacking. Attention was also paid to creating a modular architecture enabling easy expansion of the application with additional
processing techniques and research functionalities in the future. Performance benchmarks of the application and quality tests of the output files were carried out based on real astronomical data. The analysis of the obtained results confirmed the correctness
of the implementation and usefulness of the application in the context of amateur astrophotography. The thesis also presents proposals for the usage of the implemented techniques and directions for further development as well as research in the future.