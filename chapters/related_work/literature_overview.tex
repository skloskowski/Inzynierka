\section{Przegląd literatury}

Literatura związana z przetwarzaniem obrazów astronomicznych, niezbędna do wykonania pracy, obejmuje charakterystykę szumu i zależności stosunku sygnału do szumu od czasu integracji oraz źródeł
szumu w badanych aparatach DSLR oraz układach CCD używanych w astrofotografii. Do wymaganych podstaw teoretycznych odniesiono się do prac \cite{Hainaut2005,Howell2006}. W pozycji \cite{Howell2006} opisano
również metody kalibracji obrazów astronomicznych przy użyciu dedykowanych klatek kalibracyjnych oraz ich wpływ na jakość obrazu wynikowego. Informacje dotyczące procesu rejestracji obrazów astronomicznych
oraz dopasowania transformacji afinicznej pomiędzy obrazami zostały zaczerpnięte z prac \cite{rs15071921,doi:10.2514/6.2025-99706}

Do uzyskania informacji na temat warunków wykonywania zdjęć astronomicznych wykorzystano prace \cite{DTIC2001,Bortle2006,Cheremkhin_2014}, które opisują wpływ zanieczyszczenia 
świetlnego, warunków atmosferycznych na jakość uzyskiwanych obrazów oraz informacje dotyczące charakterystyki spektralnej aparatów DSLR. Informacje dotyczące obiektów astronomicznych zaczerpnięto 
z katalogu NGC \cite{ngc}. Dodatkowo posłużono się źródłami udostępniającymi mapy zanieczyszczenia świetlnego \cite{lightpollutionmap} oraz informacji dotyczącnych wartości szumu dla różnych 
ustawień aparatu dostępnych na stronie \cite{photonstophotos}.

W trakcie implementacji aplikacji wykorzystano dostępne informacje na temat używanych w środowisku algorytmów nakładania obrazów astronomicznych z dostępnych narzędzi przetwarzania obrazów 
astronomicznych takich jak DeepSkyStacker \cite{DeepSkyStacker}, Siril \cite{Siril} oraz PixInsight \cite{PixInsight}. Dodatkowe informacje na temat algorytmu Auto Adaptive Weighted Average zaczerpnięto
z \cite{Stetson1994}. Do implementacji wykorzystano również dostępne biblioteki OpenCV \cite{opencv} oraz LibRaw \cite{libraw} umożliwiajace kolejno przetwarzanie obrazów oraz odczyt plików RAW.