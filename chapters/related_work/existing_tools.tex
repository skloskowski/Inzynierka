\section{Istniejące narzędzia}

Na rynku jest dostępnych wiele narzędzi przeznaczonych do obróbki zdjęć obiektów głębokiego nieba. Wiele z nich oferuje wiele zaawansowanych funkcjonalności, ponieważ zostały stworzone z myślą o doświadczonych użytkownikach. 
Obsługa tych programów bywa skomplikowana, a poznawanie interfejsu użytkownika jest czasochłonne. Poniżej zostaną przedstawione najpopularniejsze z programów używane do obróbek astrofotografii.

\begin{itemize}
    \item \textbf{DeepSkyStacker}

    DeepSkyStacker \cite{DeepSkyStacker} jest jednym z najbardziej popularnych, darmowych narzędzi przeznaczoncych do łączenia zdjęć obiektów głębokiego nieba. Pozwala na automatyczne wyrównania klatek, kalibrację przy użyciu klatek kalibracyjnych i 
    łączenie obrazów różnymi metodami. Nie umożliwia jednak użytkownikowi na przetwarzanie zdjęć planet czy bezpośrednią obróbkę w aplikacji. Program oferuje graficzny interfejs użytkownika, który przez dużą ilość swoich możliwości może być mało 
    intuicyjny dla początkujących.
    
    \item \textbf{PixInsight}

    PixInsight \cite{PixInsight} to zaawansowane komercyjne narzędzie używane przez profesjonalistów. Zapewnia bardzo szerokie możliwości przetwarzania danych, kalibracji i analizy fotometrycznej. Głównymi wadami programu jest jej cena oraz 
    duża ilość wymaganej wiedzy do prawidłowej obsługi programu. 
    
    \item \textbf{Siril}

    Siril \cite{Siril} to darmowe i wieloplatformowe narzędzie do przetwarzania astrofotografii. Oferuje prostszy interfejs niż PixInsight. Jego główną zaletą, dla bardziej zaawansowanych użytkowników, jest możliwość importowania 
    skryptów do przetwarzania fotografii; dla początkujących fotografów jest to niepotrzebne i zwiększa tylko wymaganą wiedzę do korzystania z podstawowych funkcji programu. 

\end{itemize}