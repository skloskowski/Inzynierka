\section{Interfejs użytkownika}\label{chapter:user_interface}

Prosty i czytelny interfejs użytkownika jest wymaganą częścią programów zaprojektowanych dla użytkowników początkujących. Poniżej został opisany interfejs użytkownika wraz z krótką instrukcją użytkowania.

Interfejs użytkownika został zaprojektowany upraszczając interakcję pomiędzy użytkownikiem a programem, nie pozbawiając go jednak podstawowych funkcjonalności programu. Na potrzeby obecnego działania programu zdecydowano się na wykorzystanie interfejsu konsolowego. Pozwoliło to na skupienie się na prawidłowym działaniu najważniejszych funkcjonalności programu. 

Funkcję programu podzielone na trzy główne elementy:

\begin{itemize}
    \item Inicjalizacja obszaru roboczego.
    \item Tworzenie klatek kalibracyjnych.
    \item Rejestracja oraz stacking zdjęć obiektów głębokiego nieba.
\end{itemize}

Proces inicjalizacji programu polega na utworzeniu wykorzystywanej przez program struktury katalogów w wybranym folderze roboczym. Do inicjalizacji stosuje się poniższe polecenie.

\begin{verbatim}
.\FluxStack --init <workspace>
\end{verbatim}

W procesie kalibracji zdjęć łączy się dostarczone przez użytkownika fotografie kalibracyjne w klatki główne. Po wykonaniu tego procesu w katalogu \texttt{<workspace>\textbackslash masters} pojawią się, odpowiednio dla wybranych flag, gotowe klatki kalibracyjne. W przypadku braku flag program spróbuje utworzyć wszystkie trzy główne klatki kalibracyjne. Utworzenie klatek kalibracyjnych nie jest wymagane w dalszych etapach działania programu, przynoszą one jednak znaczne korzyści dla końcowej jakości wygenerowanego zdjęcia. Do kalibracji używa się poniższe polecenie. Flagi \texttt{-d -b -f} odnoszą się kolejno do utworzenia klatek dark, bias i flat. Prawidłowy sposób sporządzenia tych klatek przybliżono w rozdziale \ref{chapter:image_processing}.

\begin{verbatim}
.\FluxStack.exe --calibrate <workspace> [-d] [-b] [-f]
\end{verbatim}

Do nakładania zdjęć obiektów należy najpierw wykonać odpowiednią rejestrację zdjęć. Polega ona na wyliczeniu przesunięć pomiędzy poszczególnymi zdjęciami w katalogu \texttt{lights}, a wybranym przez program zdjęciem referencyjnym. Proces wykonuje się automatycznie
przy uruchomieniu polecenia z flagą \texttt{---stack}.
W procesie stackowania zdjęcia z katalogu \texttt{lights} są łączone w obraz wynikowy działania programu korzystając z jednego z możliwych algorytmów. Algorytm wybiera się poprzez flagę \texttt{---method}. Dostępne algorytmy to średnia (\texttt{average}), 
mediana (\texttt{median}), kappa-sigma clipping (\texttt{kappasigma}) oraz auto adaptive weighted average (\texttt{adaptive}). Bez podania flagi \texttt{---method} program automatycznie wybiera metodę średniej. Dodatkowo użytkownik ma możliwość zapisania 
pliku wynikowego w wybranej przez niego lokalizacji korzystając z flagi \texttt{---out}. Bez podania flagi \texttt{---out} obraz zapisuje się w lokalizacji \texttt{masters\textbackslash stacked.tiff}. Poniżej znajduje się schemat polecenia pozwalającego na 
stacking obrazów.

\begin{verbatim}
.\FluxStack.exe --stack <workspace> 
[--method=<average|median|kappasigma|adaptive>] [--out=<path>]
\end{verbatim}

W razie podania nieprawidłowych parametrów czy flag, bądź włączeniu programu bez żadnych flag, program przypomina użytkownikowi o możliwych funkcjach programu.



