\section{Architektura systemu}\label{chapter:system_architecture}

\begin{figure}[!ht]
    \centering
    \includegraphics[width=0.5\linewidth]{images/mainflow.png}
    \caption{Główne założenia działania programu.}
    \label{fig:mainflow}
\end{figure}

Na rysunku \ref{fig:mainflow} przedstawiono główny zarys działania programu. W celu realizacji zadań systemu oraz umożliwienia przyszłego rozwoju zdecydowano się na podział na odpowiednie,
niezależne od siebie, moduły. System składa się z modułów inicjalizacji, przetwarzania fotografii, rejestracji, kalibracji, nakładania zdjęć, 
interakcji z plikami i sterującego z elementami interakcji z użytkownikiem. 

\begin{itemize}
    \item Moduł inicjalizacji odpowiada za przygotowanie środowiska pracy. Tworzy odpowiednią strukturę katalogów składającą się z odpowiednich podkatalogów na fotografie kalibracyjne, główne zdjęcia light oraz
katalog na pliki wynikowe. Dzięki implementacji tego modułu użytkownik nie musi ręcznie tworzyć odpowiedniech podkatalogów co ułatwi interakcję z programem oraz zmniejszy ryzyko błędów 
wynikających z nieporawidłowej organizacji danych.
    \item Moduł przetwarzania fotografii odpowiada za podstawowe przygotowanie fotografii do dalszych etapów przetwarzania jak wykrywanie gwiazd oraz jest odpowiedzialny za końcową obróbkę 
    zdjęć po nałożeniu. Do odpowiedniego przygotowania plików przed wykrywaniem gwiazd należy przeprowadzić odpowiednią konwersję z danych surowych na dane z tylko jednym kanałem. Końcowa obróbka danych
    zakłada przetwarzanie zdjęć pod względem ustalenia odpowiedniego balansu bieli oraz normalizację jasności zdjęcia wynikowego.
    \item Moduł rejestracji odpowiada za lokalizację gwiazd na fotografiach oraz obliczenie odpowiedniej transformacji w celu nałożenia zdjęć. Do wykrywanie gwiazd wykorzystywana jest funkcja udostępniona
    przez bibliotekę OpenCV, która pozwala na wykrycie i ustalenie centroidów odpowiednich obiektów nazywanych jako \textit{blob}. Odpowiednie parametry pozwalające wykrywanie
    gwiazd zostały opisane w rozdziale \ref{chapter:code_structure}. Ustalenie transformacji będzie możliwe dzięki oszacowaniu transformacji afinicznej na podstawie dopasowanych punktów.
    \item Moduł kalibracji jest odpowiedzialny za przygotowanie i nakładanie zdjęć kalibracyjnych. Zdjęcia kalibracyjne obsługiwane przez program to bias, dark oraz flat.
    \item Moduł nakładania zdjęć odpowiada za łączenie przygotowanych fotografii w jeden obraz końcowy. Moduł będzie obsługiwał metody łączenia zdjęć: średnia, mediana, Kappa-Sigma Clipping oraz Auto Adaptive
Weighted Average.
    \item Moduł sterujący odpowiada za zarządzanie przepływem danych pomiędzy poszczególnymi modułami oraz interakcję z użytkownikiem.
    \item Moduł interakcji z plikami odpowiada za odczyt plików RAW przygotowanych prez użytkownika, odczytywanie z dysku odpowiednich plików przejściowych kalibracyjnych oraz zapisywanie plików wynikowych.
\end{itemize}